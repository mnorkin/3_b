\documentclass[a4paper,11pt,twocolumn]{IEEEtran}

%% Language and encoding
\usepackage[utf8x]{inputenc}
\usepackage[L7x]{fontenc}
\usepackage[lithuanian]{babel}

\author{Maksim Norkin\\Vilniaus Gedimino technikos universitetas\\Elektronikos fakultetas\\Elektroninių sistemų katedra\\\texttt{maksim.norkin@ieee.org}}
\title{Bakalauro baigiamasis darbas\\Parkinsono ligos eigos stebėjimo priemonė\\1 Plakatas}

\begin{document}

	\maketitle
	
	\section{Užduotis}
	
	Bakalauro baigiamojo darbo tema -- Parkinsono ligos eigos stebėjimo programa. Parkinsono liga yra dažniausiai pasitaikantis neurodegeneracinis judėjimo sutrikimas. Ankstyva ligos diagnozė ir efektyvus terapijos stebėjimas yra būtinas pacientų gydymui ir ligos. Šiuo metu neegzistuoja gydytojų patvirtintos objektyvios ir vieningos vertinimo sistemos, kuri tiksliai atpažintų Parkinsono ligos simptomus. Vienas iš didžiausiai pasireiškiančių simptomų yra eisenos sutrikimas. Sutrikimo dažnumą ir svarbą patvirtina viešai prieinama duomenų bazė, kurioje yra pateikiami sveikų ir sergančių Parinsono liga žmonių eisenos duomenys.
	
	Darbo tikslas -- sukurti programą, kuri atpažintų sveiką ir Parkinsono liga sergantį žmogų, remiantis vertikalios jėgos jutikliais gautais signalais. Programa įgyvendinta Matlab platformoje. Ji pasirinkta dėl plataus įrankių kiekio, kuris yra įgyvendintas Matlab aplinkoje. Nurodytoje platformoje taip pat yra labai patogu ir greita realizuoti signalų apdorojimo sistemas dėl jos architektūros -- visi kintamieji yra matricos. 
	
	Sukurtas produktas gebės pateikti diagnozės rezultatą -- ar subjektas turi Parkinsono liga sergančių subjektų eisenos požymių ar jų neturi. Produktas neatsižvelgs į kitus ligos simptomus: drebulys (rankų, kojų, žandikaulio, galvos), standumas (galūnių arba liemens sustingimas), bradikinezija (judesių lėtumas), pozicijos nestabilumas (arba sutrikęs balansas). Pati programa duomenis analizuos jau po duomenų surinkimo. Tai reiškia, kad pirmiausiai duomenys yra surenkami, o vėliau įkeliami į programą tolimesniam apdorojimui.
	
	\section{Rezultatai}
	
	Darbo metu ištirtos galimos žingsnio savybės, kuriomis remiantis galima sėkmingai atpažinti Parkinsono ligą pagal subjekto eiseną. Nustatyta, kad dažninės žingsnio komponentės, koreliacijos koeficientas, dviejų maksimumų ir vieno minimumo savybės neturi pakankamai informacijos Parkinsono ligos atpažinimui. Daugiausiai informacijos turi kojos prisilietimo prie žemės ir kojos pakilimo nuo žemės signalo laiko variacijos.

Turint duomenis, kurie turi daugiausiai informacijos ligos identifikavimui, toliau patikrinta galimų dimensijų praskyrimo metodų pritaikymas. Linijiniai PCA ir LDA transformacijos savybių erdvę duomenys tinkamai nepraskyrė. Geriausiai užduotį LDA su Gauso branduoliu. Rezultatas parodė kaip branduolio metodo pritaikymas gali padidinti dimensijų mažinimo algoritmo efektyvumą.

	Turimus vienmačius duomenis realiu laiku geriausiai klasifikavo naivus Bayes klasifikatorius, tačiau klasifikavimo tikslumas ir taiklumas nepatenkinamas, tikslumas siekė tik $50,8~\%$. Pritaikius papildomą metodikos žingsnį -- laikinosios atminties bloką, kuriame saugomas pirmas transformacijos diskriminantas, klasifikavimo tikslumas pagerintas iki $80~\%$. Rezultatas nurodo kas penkto diagnozavimo rezultato klaidingumą. Turint omenyje, kad klinikinės diagnostikos sprendimų taiklumas yra nuo $74~\%$ iki $90~\%$, sistemos darbo rezultatas kitų produktų palyginime atrodo patenkinamai.
	
	Darbe panaudotos priemonės lūkesčius pateisino dalinai. Išskirta savybė identifikuoja ligos požymius tik esant ilgos eisenos prielaidai -- subjektas turi atlikti eisenos patikrą ilgiau negu $2~min$. Rekomenduojama eisenos trukmė yra $5~min$. Tokiu atveju sistema geriausiai identifikuos ligos simptomą pagal parinktą savybę. Taip pat klasifikavimo rezultatus gali pagerinti abiejų kojų naudojimas savybių išskyrimo metu, kadangi eisenos nesimetriškumas gali galioti ne tik kairiai kūno pusei, tačiau ir dešinei.

\end{document}